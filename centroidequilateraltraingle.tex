\documentclass[oneside]{book}
\usepackage{braket}
\usepackage[latin1]{inputenc}
\usepackage{amsfonts}
\usepackage{amsthm}
\usepackage{amsmath}
\usepackage{mathrsfs}
\usepackage{enumitem}
\usepackage[pdftex]{color,graphicx}
\usepackage{hyperref}
\usepackage{listings}
\usepackage{calligra}
\usepackage{algpseudocode} 
\DeclareFontShape{T1}{calligra}{m}{n}{<->s*[2.2]callig15}{}
\newcommand{\scripty}[1]{\ensuremath{\mathcalligra{#1}}}
\setlength{\oddsidemargin}{0cm}
\setlength{\textwidth}{490pt}
\setlength{\topmargin}{-40pt}
\addtolength{\hoffset}{-0.3cm}
\addtolength{\textheight}{4cm}
\usepackage{amssymb}
\usepackage{graphicx} % Required for the inclusion of images
\setlength\parindent{0pt} % Removes all indentation from paragraphs
\usepackage{float}
\usepackage{makeidx}
%\begin{figure}[H]
%	\centering
%	\includegraphics[scale = 0.42]{lcaoderecha}
%	\caption{Eletr\'on ligado solo al n�cleo derecho}
%	\label{fig1}
%	\end{figure}




\begin{document}
%\tableofcontents
%\pagebreak










\begin{center}
\textsc{\LARGE Centroide de una Placa Uniforme en Forma de Triangulo Equilatero }\\[0.5cm]
\end{center}


\textbf{M�todo 1: Integral Doble }\\


Tome el marco de referencia en la mitad de uno de los lados del triangulo equilatero. Se observa que la regi�n de integraci�n est� definida por las siguientes ecuaciones:


\begin{eqnarray}
\label{1}   -\frac{a}{2} \leq x \leq \frac{a}{2}
\end{eqnarray}

\begin{equation}
\label{2}  0 \leq y \leq \left(\frac{a}{2} - |x|\right) \tan 60
\end{equation}


La integral para el centro de masa en la posici�n $y$ es dada por:


\begin{equation}
\label{3} \bar{y} = \frac{1}{A} \int_{-a/2}^{a/2} \int_{0}^{ \left(\frac{a}{2} - |x|\right) \tan 60}   y\; dy dx
\end{equation}


\begin{equation}
\label{4}  \int_{-a/2}^{a/2} \int_{0}^{ \left(\frac{a}{2} - |x|\right) \tan 60}   y\; dy dx = \frac{1}{2} \int_{-a/2}^{a/2} \left(\frac{a}{2} - |x|\right)^2 (\tan 60)^2 dx = \frac{3}{2} \int_{-a/2}^{a/2} \left(\frac{a}{2} - |x|\right)^2  dx
\end{equation}


Donde se uso $(\tan 60)^2 = 3$. Se observa que el integrando es una funci�n par por lo que se tiene:



\begin{equation}
\label{5}  \frac{3}{2} \int_{-a/2}^{a/2} \left(\frac{a}{2} - |x|\right)^2  dx = 3 \int_{0}^{a/2} \left(\frac{a}{2} - |x|\right)^2  dx =  3 \int_{0}^{a/2} \left(\frac{a}{2} - x\right)^2  dx
\end{equation}

Usando el cambio de variable $u = \frac{a}{2} - x$ se obtiene:


\begin{equation}
\label{6}  3 \int_{0}^{a/2} \left(\frac{a}{2} - x\right)^2  dx =  3 \int_{0}^{a/2} u^2 du = \frac{a^3}{8}
\end{equation}

Por lo que finalmente se obtiene el centro de masa dividiendo por el �rea que es $A = \frac{\sqrt{3}}{4}a^2$:


\begin{equation}
\label{7} \bar{y} = \frac{a}{2 \sqrt{3}}
\end{equation}

Para la posici�n en $x$:



\begin{equation}
	\label{9} \bar{x} = \frac{1}{A} \int_{-a/2}^{a/2} \int_{0}^{ \left(\frac{a}{2} - |x|\right) \tan 60}   x\; dy dx = \frac{1}{A} \int_{-a/2}^{a/2}   \left(\frac{a}{2} - |x|\right)  x \tan 60\;  dx = 0
\end{equation}

Donde se uso que la ultima integral tiene integrando impar.\\

\textbf{M�todo 2: Una sola Integral}\\


El problema se puede pensar como sumar rect�ngulos de ancho infinitesimal. El �rea de cada rect�ngulo es $dA = 2x' dy$, donde $x'$ es dado por $x'= a/2 - x$ y $x' \tan 60 = y$. Es decir $x'$ es una distancia sobre una cara del triangulo equilatero, medida desde un v�rtice. $x$ es la distancia del centro de esa cara del triangulo al punto $x'$. Entonces $2x$ es el largo de nuestros rect�ngulos a integrar. La altura corre desde $0$ hasta la altura del triangulo, $\sqrt{3}a/4$. Por lo que la doble integral se reduce a la siguiente simple integral:  


\begin{equation}
\label{8} \bar{y} = \frac{1}{A} \int \int y \;dA = \frac{1}{A} \int_{0}^{\frac{\sqrt{3}}{2}a}  \left( a - \frac{2 y}{\tan 60} \right) y \; dy = \frac{a}{2\sqrt{3}}
\end{equation}


\end{document}