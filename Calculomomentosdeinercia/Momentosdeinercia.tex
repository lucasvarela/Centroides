%%% Template originaly created by Karol Kozioł (mail@karol-koziol.net) and modified for ShareLaTeX use

\documentclass[a4paper,11pt]{article}

\usepackage[T1]{fontenc}
\usepackage[utf8]{inputenc}
\usepackage{graphicx}
\usepackage{xcolor}
\usepackage{parskip}
\usepackage{tgtermes}

\usepackage[
pdftitle={Math Assignment}, 
pdfauthor={Lucas Varela, Universidad de los Andes},
colorlinks=true,linkcolor=blue,urlcolor=blue,citecolor=blue,bookmarks=true,
bookmarksopenlevel=2]{hyperref}
\usepackage{amsmath,amssymb,amsthm,textcomp}
\usepackage{enumerate}
\usepackage{multicol}
\usepackage{tikz}
\usepackage{caption}
\usepackage{geometry}
\geometry{total={210mm,297mm},
left=25mm,right=25mm,%
bindingoffset=0mm, top=20mm,bottom=20mm}


\linespread{1.0}

\newcommand{\linia}{\rule{\linewidth}{0.5pt}}

% custom theorems if needed
\newtheoremstyle{mytheor}
    {1ex}{1ex}{\normalfont}{0pt}{\scshape}{.}{1ex}
    {{\thmname{#1 }}{\thmnumber{#2}}{\thmnote{ (#3)}}}

\theoremstyle{mytheor}
\newtheorem{defi}{Definition}

% my own titles
\makeatletter
\renewcommand{\maketitle}{
\begin{center}
\vspace{2ex}
{\huge \textsc{\@title}}
\vspace{1ex}
\\
\linia\\
\@author \hfill \@date
\vspace{4ex}
\end{center}
}
\makeatother
%%%

% custom footers and headers
\usepackage{fancyhdr,lastpage}
\pagestyle{fancy}
\lhead{}
\chead{}
\rhead{}
\lfoot{Centroides y momentos de inercia}
\cfoot{}
\rfoot{Page \thepage\ /\ \pageref*{LastPage}}
\renewcommand{\headrulewidth}{0pt}
\renewcommand{\footrulewidth}{0pt}
%

%%%----------%%%----------%%%----------%%%----------%%%

\begin{document}

\title{ Centros de Masa \& Momentos de Inercia}

\author{Complementaria de Física I}

\date{ Universidad de los Andes}

\maketitle

\section{Centros de Masa}

\textbf{Partículas puntuales:}


\begin{figure}[h]
	\includegraphics[width=0.5\linewidth]{masaspuntuales}
	\label{fcN4}
\end{figure}

El centro de masa para un sistema discreto de $N$ partículas puntuales  está dado por:

$$ \vec{r}_{cm} = \frac{1}{M_{T}}\sum_{k=1}^{N} m_k \vec{r}_k = \frac{m_1\vec{r}_1 + \dots + m_N \vec{r}_N}{m_1 + \dots + m_N}$$


Donde $ \vec{r}_k$ es la posición de la k-ésima partícula y $m_k$ su masa. $M_T$ es la suma de todas las masas.\\

\textbf{Distribución continua:}

\begin{figure}[h]
	\includegraphics[width=0.5\linewidth]{3d}
	\label{fcN4}
\end{figure}

El centro de masa de un objeto con distribución de masa continua viene dado por la siguiente integral:

$$  \vec{r}_{cm} = \frac{1}{M_T} \int \vec{r} dm $$

Donde $M_T$ es la masa total del objeto. Existen tres casos a considerar: objetos de 1, 2 y 3 dimensiones. Para cada una respectivamente existe una densidad asociada:

\begin{align*}
\lambda &= \frac{d m}{dx} \\
\sigma &= \frac{d m}{dA} \\
\rho &= \frac{d m}{dV} 
\end{align*}

Donde $\lambda$ es la densidad lineal, $\sigma$ la densidad superficial y $\rho$ la densidad (se omite decir densidad volumétrica, se le dice solo densidad usualmente). Con ellas se puede obtener las siguientes expresiones para el centro de masa de distribuciones continuas:

\begin{align*}
\vec{r}_{cm} &= \frac{1}{M_T} \int \vec{r} \lambda \;dx  \\
\vec{r}_{cm} &= \frac{1}{M_T} \int \vec{r} \sigma \;dA \\
\vec{r}_{cm} &= \frac{1}{M_T} \int \vec{r} \rho\; dV 
\end{align*}


\textbf{Sistema de múltiples distribuciones continuas:}\\



\begin{figure}[h]
	\includegraphics[width=0.5\linewidth]{muchos3d}
	\label{fcN4}
\end{figure}
El centro de masa para un sistema de $N$ objetos de distribución continua está dado por:

$$ \vec{r}_{cm} = \frac{1}{M_{T}}\sum_{k=1}^{N} m_k \vec{r}_k = \frac{m_1\vec{r}_1 + \dots + m_N \vec{r}_N}{m_1 + \dots + m_N}$$


Donde $ \vec{r}_k$ es el centro de masa de el k-ésimo elemento y $m_k$ su masa. $M_T$ es la suma de todas las masas.


\subsection{Lamina de densidad uniforme} 

Se quiere encontrar la posición del centro de masa de la lamina mostrada en la figura. La masa total
de la lamina está repartida de manera uniforme en toda su superficie. 
\begin{figure}[h]
	\includegraphics[width=0.7\linewidth]{4}
	\label{fcN4}
\end{figure}


Para ello se utiliza la siguiente estrategia: se encuentra el centro de masa de dos rectángulos.

\begin{figure}[h]
	\includegraphics[width=0.7\linewidth]{recdiv}
	\label{fcN4}
\end{figure}

Ahora, el centro de masa de cada rectángulo con respecto a ejes que estén en sus respectivos vértices inferiores izquierdos(ver figura) corresponde a la mitad de la altura y la mitad del ancho de cada uno.

\begin{figure}[h]
	\includegraphics[width=0.7\linewidth]{vert}
	\label{fcN4}
\end{figure}

Entonces el del rectangulo azul es:

$$ \vec{r}_a = \frac{b-d}{2} \hat{i} + \frac{a}{2} \vec{j}  $$

El del rectangulo verde con respecto a las coordenadas rojas es:

\color{red}

$$ \vec{r}_v = \frac{d}{2} \hat{i} + \frac{c}{2}\hat{j}$$

\color{black}
Ahora queremos saber estas posiciones con respecto al mismo sistema de coordenadas. Vemos que las coordenadas rojas estan relacionadas a las coordenadas negras sumando el vector $(b-d) \hat{i}$. Por lo tanto la coordenada de centro de masa del rectangulo verde en el sistema de coordenadas negro es:

$$ \vec{r}_v = \color{red} \vec{r}_v \color{black}+ (b-d)\hat{i}$$

Es decir:

\begin{align*}
 \vec{r}_v &= \frac{d}{2} \hat{i} + \frac{c}{2}\hat{j} +  (b-d)\hat{i}\\
 &=  \frac{2b-d}{2}\hat{i} + \frac{c}{2}\hat{j} 
\end{align*}


Ahora el centro de masa de la figura entera esta dado por:

$$ \vec{r}_{cm} = \frac{ M_v \vec{r}_v+ M_a \vec{r}_a}{M_v + M_a}$$

Donde $M_v$ y $M_a$ son las masas de los rectángulos verde y azul respectivamente. Como sabemos que la densidad es uniforme, sabemos que ambos rectangulos tienen la misma densidad superficial $\sigma$. Por lo tanto se cumple que:

\begin{align*}
M_v &= \sigma A_v\\
M_a &= \sigma A_a
\end{align*}

Donde $A_v$ y $A_a$ son las áreas de los rectangulos verde y azul respectivamente. Por lo tanto el centro de masa esta dado por:


\begin{align*}
 \vec{r}_{cm} &= \frac{ \sigma A_v \vec{r}_v+ \sigma A_a \vec{r}_a}{\sigma A_v + \sigma A_a} \\
 & = \frac{  A_v \vec{r}_v+  A_a \vec{r}_a}{ A_v +  A_a}
\end{align*}


Las áreas respectivas son:

\begin{align*}
A_v & = dc\\
A_a & = (b-d)a
\end{align*}

Por lo que el centro de masa se convierte en:

\begin{align*}
\vec{r}_{cm}  &= \frac{  dc \vec{r}_v+  (b-d)a \vec{r}_a}{ dc + (b-d)a}\\
&= \frac{  dc ( \frac{2b-d}{2}\hat{i} + \frac{c}{2}\hat{j})+  (b-d)a (\frac{b-d}{2} \hat{i} + \frac{a}{2} \vec{j})}{ dc + (b-d)a}\\
& =\frac{1}{2} \frac{  ((2db-d^2)c +a(b-d)^2) \hat{i} + (dc^2+(b-d)a^2)\hat{j}}{ dc + (b-d)a}\\
& = \frac{1}{2} \frac{   ((2db-d^2)c +a(b-d)^2) \hat{i} + (dc^2+ba^2-da^2)\hat{j}}{ dc + (b-d)a}
\end{align*}

Es decir:


\begin{align*}
X_{cm}  &= \frac{1}{2} \frac{   (2db-d^2)c +a(b-d)^2 }{ dc + (b-d)a}\\
Y_{cm}&= \frac{1}{2} \frac{   (dc^2+ba^2-da^2)\hat{j}}{ dc + (b-d)a}\\
\end{align*}

\section{Momentos de inercia}


\textbf{Partículas puntuales:}


\begin{figure}[h]
	\includegraphics[width=0.4\linewidth]{rot1}
	\caption{Eje negro y las distancias en morado de las partículas al eje.}
	\label{fcN4}
\end{figure}


\begin{figure}[h]
	\includegraphics[width=0.5\linewidth]{rot2}
	\caption{Eje negro y las distancias en morado de las partículas al eje.}
	\label{fcN4}
\end{figure}
El momento de inercia con respecto a un eje de rotación para un sistema discreto de $N$ partículas puntuales  está dado por:

$$ I = \sum_{k=1}^{N} m_k r_k^2 = m_1 r_1^2 + \dots + m_N r_N^2$$


Donde $ r_k$ es la distancia de la k-ésima partícula al eje y $m_k$ su masa. \\

\textbf{Distribución continua:}

\begin{figure}[h]
	\includegraphics[width=0.4\linewidth]{3d}
	\label{fcN4}
\end{figure}

El momento de inercia con respecto a un eje de rotación de un objeto con distribución de masa continua viene dado por la siguiente integral:

$$  I=  \int r^2 dm $$

Existen tres casos a considerar: objetos de 1, 2 y 3 dimensiones. Para cada una respectivamente existe una densidad asociada:

\begin{align*}
\lambda &= \frac{d m}{dx} \\
\sigma &= \frac{d m}{dA} \\
\rho &= \frac{d m}{dV} 
\end{align*}

Donde $\lambda$ es la densidad lineal, $\sigma$ la densidad superficial y $\rho$ la densidad (se omite decir densidad volumétrica, se le dice solo densidad usualmente). Con ellas se puede obtener las siguientes expresiones para el centro de masa de distribuciones continuas:

\begin{align*}
I &=  \int r^2 \lambda \;dx  \\
I &=  \int r^2 \sigma \;dA \\
I &=  \int r^2 \rho\; dV 
\end{align*}


\subsection{Anillo de densidad uniforme}

\begin{figure}[hh]
	\includegraphics[width=0.4\linewidth]{anillo1}
	\label{fcN4}
\end{figure}

Para calcular el momento de inercia de un anillo con respecto a un eje que pasa por el centro del anillo, perpendicular al plano donde este se encuentra, se debe hacer la siguiente integral:

$$ I = \int r^2 dm$$




El elemento de masa lo podemos expresar en términos del "volumen"(entre comillas ya que como el anillo es una figura de 1D, se considera su longitud y no volumen):

$$ dm = \lambda dr $$

Donde $\lambda$ es la densidad lineal dada por:

$$ \lambda = \frac{M}{2 \pi R } $$

\begin{figure}[hh]
	\includegraphics[width=0.4\linewidth]{anillo2}
	\label{fcN4}
\end{figure}

Y $dr$ es la posición del segmento infinitesimal del anillo que se está integrando. Como la posición de los segmentos siempre está a una distancia fija, esta posición se puede escribir en términos de un ángulo como:

$$ dr = R d\theta$$

Para recorrer todo el anillo, se debe dar una vuelta entera. Es decir $\theta$ va desde 0 hasta $2\pi$. Esto define los limites de la integral. Además se nota que la distancia desde el eje de rotación hasta cualquier parte del anillo siempre es $r=R$, por lo que se tiene:


$$ I = \int^{2\pi}_{0} \lambda  R^2 R d\theta$$

Como $\lambda$ y $R$ son constantes salen de la integral y se tiene:

$$ I = R^3 \lambda \int_{0}^{2\pi} d\theta = R^3 \lambda 2\pi = m R^2 $$

\subsection{Disco de densidad uniforme}


\begin{figure}[h]
	\includegraphics[width=0.5\linewidth]{disc1}
	\label{fcN4}
\end{figure}

Se va a calcular el momento de inercia de un disco de masa $M$ distribuida uniforme. Este momento de inercia sera con respecto al eje que atraviesa el disco por el centro de masa (el centro) y perpendicular al plano del disco.


\begin{figure}[h]
	\includegraphics[width=0.5\linewidth]{disc2}
	\label{fcN4}
\end{figure}


Para calcular el momento de inercia del disco vamos a dividir el disco en anillos y sumar la contribución de cada anillo. Estos anillos tiene un grosor infinitesimal $dy$. Ya vimos que el momento de inercia de un anillo es:

$$ dI =  r^2 dm_a $$


\begin{figure}[h]
	\includegraphics[width=0.5\linewidth]{disc3}
	\label{fcN4}
\end{figure}

Donde $dm_a$ es la masa de cada anillo que vamos a sumar y $r$ el radio de cada anillo. La integral se encarga de sumar cada anillo con grosor infinitesimal. Por lo tanto el momento de inercia esta dado por:


$$ I = \int d I  = \int r^2 dm_a $$

Para integrar cambiamos $dm_a$ usando la siguiente relación:


$$ dm_a = \sigma dA$$

Donde $dA$ es el área de cada anillo infinitesimal. Vemos que el área de cada anillo esta dado por la longitud del anillo por el grosor del anillo, es decir:

$$ dA = L dx = 2 \pi r d r$$

 Juntando estos resultados el momento de inercia está dado por la siguiente expresión:
 
$$ I = \int \sigma r^2 2 \pi r dx$$



Recordando que la integral suma sobre todos los anillos, tenemos que variar $r$ desde el origen hasta el extremo del disco. Es decir que $r$ debe ir desde $0$ hasta $R$. Estos son entonces los limites de la integral:


$$ I =  \int_{0}^{R} \sigma 2\pi   r^3 dr$$

Como $\sigma$ y $2\pi$ son constantes, salen de la integral y se tiene:

$$ I =  2\pi\sigma \int_{0}^{R}    r^3 dr$$

Esta integral es:



$$ I =  2\pi\sigma \left( \frac{1}{4} R^4\right)$$

La densidad superficial $\sigma$ es:

$$ \sigma = \frac{m}{A} = \frac{m}{\pi R^2} $$


Por lo que el momento de inercia termina siendo:

$$ I = \frac{1}{2} m R^2$$


\subsection{Cilindro de densidad uniforme}



\begin{figure}[h]
	\includegraphics[width=0.5\linewidth]{cilindro1}
	\label{fcN4}
\end{figure}

Se va a calcular el momento de inercia de un cilindro de masa $M$ distribuida uniforme. Este momento de inercia se calcula con respecto a un eje que atraviesa el cilindro por el centro de masa (el centro) y perpendicular a las caras circulares del cilindro.



\begin{figure}[h]
	\includegraphics[width=0.5\linewidth]{cilindro2}
	\label{fcN4}
\end{figure}

Para calcular el momento de inercia del disco vamos a dividir el cilindro en discos y sumar la contribución de cada disco. Estos discos tienen un grosor infinitesimal $dy$. Ya vimos que el momento de inercia de un disco es:

$$ dI = \frac{1}{2} r^2 dm_d $$

Donde $dm_d$ es la masa de cada disco que vamos a sumar y $r$ el radio de cada disco. Como es un cilindro todos los discos tienen el mismo radio $R$. La integral se encarga de sumar cada disco con grosor infinitesimal. Por lo tanto el momento de inercia esta dado por:


$$ I = \int d I  = \frac{1}{2} \int R^2 dm_a $$

Como $R$ es una constante sale de la integral y se tiene:

$$ I = \frac{1}{2} R^2 \int dm_a$$

La integral $\int dm_a$ es la suma de la masa de todos los anillos, la cual es la masa total del cilindro $M$. Por lo tanto el momento de inercia es:

$$ I = \frac{1}{2} M R^2$$

Es el mismo que el del disco, solo que con la masa del cilindro. 


\subsection{Bola de densidad uniforme}


\begin{figure}[h]
	\includegraphics[width=0.5\linewidth]{bola1}
	\caption{Bola y eje de rotación}
	\label{fcN4}
\end{figure}

Se va a calcular el momento de inercia de una bola de densidad uniforme con respecto a un eje que pasa por el centro de masa (el centro) de la bola.


\begin{figure}[h]
	\includegraphics[width=1\linewidth]{bola}
	\caption{Descomposición de bola en cilindros}
	\label{fcN4}
\end{figure}

Para calcular el momento de inercia de una bola vamos a dividir la bola en cilindros y sumar la contribución de cada cilindro. Estos cilindros tiene un grosor infinitesimal $dy$. Ya vimos que el momento de inercia de un cilindro es:

$$ dI = \frac{1}{2} r^2 dm_c $$

Donde $dm_c$ es la masa de cada cilindro que vamos a sumar y $r$ el radio de cada cilindro. La integral se encarga de sumar cada cilindro infinitesimal. Por lo tanto el momento de inercia esta dado por:

$$ I = \int d I  = \frac{1}{2}\int r^2 dm_c $$




 Para integrar cambiamos $dm_c$ usando la siguiente relación:


$$ dm_c = \rho dV$$



\begin{figure}[h]
	\includegraphics[width=0.7\linewidth]{elmentovolbola}
	\caption{Cilindro infinitesimal}
	\label{fcN4}
\end{figure}

Donde $dV$ es el volumen de cada cilindro infinitesimal. Vemos que el volumen de cada cilindro esta dado por el área del disco por el grosor cilindro, es decir:

$$ dV = A dy = \pi r^2 d y$$

 Juntando estos resultados el momento de inercia está dado por la siguiente expresión:
 
 $$ I = \frac{1}{2} \int \rho r^2 \pi r^2 dy$$

Ahora, para poder integrar esta expresión, tenemos que encontrar como depende el radio de los cilindros en función de la altura $y$. Usando el teorema de Pitágoras tenemos que:

$$ r^2 = R^2 - y^2$$

Por lo tanto el momento de inercia está dado por:

 $$ I = \frac{1}{2} \int \rho \pi   (R^2-y^2)^2 dy$$

Recordando que la integral suma sobre todos los cilindros, tenemos que variar $y$ desde la parte inferior de la bola hasta la superior. Es decir que $y$ debe ir desde $-R$ hasta $R$. Estos son entonces los limites de la integral:


$$ I = \frac{1}{2} \int_{-R}^{R} \rho \pi   (R^2-y^2)^2 dy$$

Ahora lo único que resta es hacer la integral. Primero, notamos que como la densidad y el número $\pi$ son constantes (no depende de $y$), salen de la integral:


$$ I = \frac{1}{2} \rho \pi \int_{-R}^{R}    (R^2-y^2)^2 dy$$

Ahora simplificando la expresión $(R^2-y^2)^2$:

$$ (R^2-y^2)^2 = R^4 -2 R^2 y^2 + y^4$$

Por lo que:


$$ I = \frac{1}{2} \rho \pi \int_{-R}^{R}  R^4 -2 R^2 y^2 + y^4   dy$$

Ahora usamos la propiedad de que la integral de una suma es la suma de las integrales:
$$ I = \frac{1}{2} \rho \pi \left(  R^4\int_{-R}^{R} dy -2 R^2 \int_{-R}^{R} y^2 dy + \int_{-R}^{R}y^4   dy\right)$$

La integral de un monomio esta dada por:

$$ \int_{a}^{b} x^{n} dx = \left.\frac{x^{n+1}}{n+1}\right|_{a}^{b} = \frac{b^{n+1}}{n+1} - \frac{a^{n+1}}{n+1}$$


Por lo que el momento de inercia es:


\begin{align*}
I &= \frac{1}{2} \rho \pi \left(  R^4(R-(-R))  -2 R^2 \left( \frac{R^3}{3}-\frac{(-R)^3}{3}\right)  + \left( \frac{R^5}{5}-\frac{(-R)^5}{5}\right)\right) \\
I  &=\frac{1}{2} \rho \pi \left(  2R^5 -2 R^5 \frac{2}{3}  + R^5 \frac{2}{5}\right)\\
I & = \frac{1}{2} \rho \pi R^5 \frac{16}{15}\\
I & = \frac{8}{15} \rho \pi R^5
\end{align*} 

La densidad de una bola es:

$$ \rho = \frac{M}{V} = \frac{M}{
	\frac{4}{3} \pi R^3}$$

Por lo que finalmente se tiene que el momento de inercia de una bola es:

$$ I= \frac{8}{15} \frac{M}{
	\frac{4}{3} \pi R^3} \pi R^5 =  \frac{2}{5} M R^2$$


Calcule el momento de inercia de un disco con distribución de masa uniforme de radio $R$ con respecto al eje perpendicular al disco que pasa por su centro de masa. 




\section{Teorema de ejes paralelos}

El teorema de ejes paralelos da un formula para calcular el momento de inercia de un sistema con respecto a un eje de rotación que este paralelo a otro eje de rotación que pase por el centro de masa del objeto. Si $I_{cm}$ es el momento de inercia con respecto al eje que pasa por el centro de masa, y $a$ es la distancia entre dichos ejes se tiene que el momento de inercia del eje paralelo viene dado por:

 $$ I_{a} = I_{cm} + M a^2$$


\subsection{Barra con eje en un extremo}

\begin{figure}[h]
	\includegraphics[width=0.4\linewidth]{paralelobarra}
	\label{fcN4}
\end{figure}

El eje de la barra se encuentra a una distancia $L/2$ de un eje que es paralelo y pasa por el centro de masa. Por lo que por el teorema de ejes paralelos se tiene que el momento de inercia esta dado por la siguiente expresión:

$$ I = I_{cm} + m \left(\frac{L}{2}\right)^2$$



El momento con respecto al eje que pasa por el centro de masa es $I_{cm} =\frac{1}{12} m L^2$. Entonces el momento de inercia es:

$$ I = \frac{1}{3} m L^2$$

\subsection{Bola con eje en un extremo}

\begin{figure}[h]
	\includegraphics[width=0.4\linewidth]{paralelobola}
	\label{fcN4}
\end{figure}

El eje de la bola se encuentra a una distancia $R$ de un eje que es paralelo y pasa por el centro de masa. Por lo que por el teorema de ejes paralelos se tiene que el momento de inercia esta dado por la siguiente expresión:

$$ I = I_{cm} + m R^2$$

El momento con respecto al eje que pasa por el centro de masa es $I_{cm} =\frac{2}{5} m R^2$. Entonces el momento de inercia es:

$$ I = \frac{7}{5} m R^2$$



\subsection{Discos}




\begin{figure}[h]
	\includegraphics[width=0.4\linewidth]{parelelodiscos}
	\label{fcN4}
\end{figure}


En este ejemplo se tiene 5 discos de masa $m$ distribuida uniforme y todos con mismo radio. Se quiere encontrar el momento de inercia con respecto a un eje que atraviesa el disco central por justo en el centro. Para hacerlo sumamos el momento de inercia de los 5 discos con respecto a dicho eje. Es decir:

$$ I = I_1 + I_2 + I_3 + I_4 + I_5$$

Donde se $I_k$ es el momento de inercia del k-ésimo disco. En el siguiente dibujo se muestra como se enumeraron los discos.
\begin{figure}[h]
	\includegraphics[width=0.6\linewidth]{numerados}
	\label{fcN4}
\end{figure}

El momento del primer disco es el de un disco con el eje que pasa por el centro de masa y es $I_1 = 1/2 m R^2$. El eje está paralelo al eje que pasa por el centro de masa del disco 2, con una distancia entre ejes de $2R$. Por lo que por el teorema de ejes paralelos:

$$ I_2 = I_{cm}  + m(2R)^2 = \frac{1}{2} mR^2 + 4 mR^2 = \frac{9}{2} m R^2$$

Para el disco 3,4 y 5 se tiene lo mismo, por lo que:

\begin{align*}
I &= I_1 + 4 I_2 \\
&= \frac{1}{2} mR^2 + 4 \left(\frac{9}{2} m R^2\right)\\
& = \frac{37}{2} m R^2 
\end{align*}



\end{document}
